\documentstyle{article}
\newcommand{\be}{\begin{equation}}
\newcommand{\ee}{\end{equation}}
\newcommand{\bv}{\begin{verbatim}}
\newcommand{\ev}{\end{verbatim}}
\begin{document}
\title{How to use the upi code}
\author{D. M. Ceperley}
\maketitle
\section{Introduction}
This document gives instructions on how to run the path integral
program.

One starts the study of a new physical simulation with the 
specification of the potential energy function. This is done
in the program POTGEN and discussed in the previous manual. 
Usually one then uses the output of
POTGEN to construct the density matrix with the SQUARER code.  
A set up program constructs of the
input to FNPIMC.  OCSET is a set up program for a single component
systems, either liquid or solid but homogenous.
Finally FNPIMC is run and the results are
analyzed with various other programs. Although I have
generic 
programs it is expected that the user modify OCSET and
the analysis codes for more their application.

Please look at the manual for Squarer for the syntax for SQUARER.

\section{Input Commands for UPI}
Commands can be in any order as long as a few rules are obeyed.
Particle types must be defined before they are used.

\begin{enumerate}

\item UNITS (optional)
\bv
UNITS eunit lunit : 
\ev
This is used for labelling various quantities but not to calculate
anything.
 
\item
  
\bv
     ENORM    x
\ev
The total energy is divided by x.
\item IFDIAG
\bv
     IFDIAG   n
\ev
Level at which off-diagonal terms are used
\item NOLEAK
\bv
     NOLEAK   
\ev
\item BOXSIZE (necessary command)
\ev
     BOXSIZE  L1 L2 Ldim
\ev
The size of the periodic box in the ndim  directions. ndim is a parameter
isn uppi.p which sets the number of physical dimensions.1
\item CUTK
\bv
     CUTK    
\ev
Cutoff in k-space for s(k).
\item BETA (necessary)
\bv
BETA     1  1.        
\ev    
where $\beta =1/(k_b*T) $.
\item NSLICES (necessary)
\bv
     NSLICES  nslices 
\ev
 number of time slices
\item FREESAMP
\bv
     FREESAMP t1 t2...  off 
\ev
If present free particle sampling, otherwise  will use the
sampling table. t1 t2 and so on are the types of particles
that you want to turn free sampling on for.
\item KSAMP
\bv
     KSAMP t1 t2... 
\ev
if present turn on k-space sampling
t1 t2 and so on are the types of particles
that you want to turn k-space sampling on for.
One must have regular sampling on for these particle and be using Ewald
sums.
\item GAMMA
\bv
     GAMMA   nmovers type value 
\ev
 prefactor for sampling permutation (SELECT) and probability 
of attempting moves in OMOVE defaults are 1
\item NREF
\bv
     NREF   n 
\ev
 number of reference points (0,1 or 2)
            0 corresponds to ground state nodes
            1 is 1 reference point
            2 is 2 symmetrically arranged points
\item WRITEPC
\bv
     WRITEPC  freq
\
The number of passes between writes to file id.pc
of the compressed configuration files. The default is to never write.
\item DEBUG
\bv
DEBUG   debug 
\ev
debug level (0 =no tests, 1 some test, 2 many tests)
Default is no tests.

\item DISPLACE
\bv
 DISPLACE  type distance [ gamma 
\ev
  move the entire chain uniformly in a cube
  of  side distance -not for fermions
  gamma is the probability of attempting individual  move
\item TIMER (optional)
Timer set the maximum time in CPU seconds that a block can run for.
\bv
TIMER  time
\ev
Without this command upi will execute until the specified number of
passes are finished.
         
\item TYPE (required)
One must have a TYPE command for each species of particles.
\bv
TYPE name nspins number hbs2m (nppss(i),i=1,nspins) fname [ charge.
\ev
     hbs2m =0 means a fixed particle

Number is the number of particles of this type.

     nspins =-1 means distinguishable particles
     nspins =0  means bosons
     nspins >0  means fermions with spin degeneracy of nspins.
     In this case the number of each spin will be nppss.

fname contains the initial starting positions.
if name starts with a '+' aged configurations will be assumed

     charge = optional charge on species
\item LINE
defines a line particle (like a vortex line). 
\bv
  LINE name x1 y1 | ...  xn | zn
\ev

The input are triplets (in 3d) or pairs (in 2d). `
The coordinates of the particles are given as (x y z)

with the number representing the orientation of the line 
being replaced by the symbol '|' (the UNIX pipe).
Example: (LINE vortex 0. 0. | ) represents are 'vortex' particle
which is a line passing through the origin in the z direction.

\item PLANE
Defines n plane-like particles.  
\bv
  PLANE name d1 x1 ... dn xn
\ev
The input are pairs (d1 x1) which means
the normals are in the d1 direction and
that coordinate is x1. where $d_i$ is an integer $1 \le d_i \le ndim$.

\item POT PAIR (optional)

This command generates an interaction between the two types of particles.
file is assumed to be generated from squarer.
\bv
POT PAIR type1 type2 file [ntact nteng
\ev
Without a POT PAIR command there will be no pair interactions.

The code saves space by using pointers to the data, thus if
'file' appears a second time, the data is not read in but
the second pointer points to the first location.  This implies
that the computed pair correlations are also grouped together.
The potential and action are assumed to be constant outside
the range of the table. 
The optional parameters ntact and nteng are the number of offdiagonal
terms used in the evaluation of the action and energy (their 
default is 3 and 6).  
 
\item  SEED (optional)
used to set the random number seed
Default is to use the clock and possibly the processor number.
The number actually used is written out and the job can be rerun by
using that as the seed.
\bv
SEED value
\ev

\item  RESTART (required)
This command seperates the setup from the running of the code.
\bv
RESTART [ type  [file
\ev
There are 3 levels of restart. NONE, PARTIAL (only pickup up 
path not averages) and FULL. default is NONE
The default for file is qid.rs
 
\item 
OMOVE nblocks , nsteps , levels
   nblocks = number of blocks to do
   nsteps = total number of passes/block
   levels = level of move (i.e. =1 only midpoint..)
       a pass consists an attempted move of all particles
       after a pass we do analysis, reference move, write out configurations
    and insertion, and a displacement.
 
\item  SELECT
\bv
SELECT nblocks , nsteps, levels, nhits
\ev
       a pass consists an nhits attempted moves
       after a pass we do analysis, reference move, write out configurations
    and insertion, and a displacement.
 
\item
\begin{verbatim}
PERMUTE nblocks , nsteps, levels, nhits
\end{verbatim}
The input is rather similar to SELECT but 'nhits' and 'gamma' must
be chosen differently.
A pass consists  of first randomly selecting a time interval,
constructing the table of pair distances, then making nhits attempted construction
of cyclic permutations.  
After a pass we do analysis, reference move, write out configurations
and insertion, and a displacement.
If the parameter TIMER is set, the block is finished after a certain amount
of CPU time, instead of a fixed number of passes.

PERMUTE differs from select in that the permutation step is accepted or
rejected before the call to WEAVE, (where the path is constructed) and
that the cyclic permutation is constructed from a random walk through label
space instead of explicitly looking at all possible cyclic permutations.
This means that one can do longer permutation moves simply by increasing
MNMOVERS.  Currently PERMUTE is less robust than SELECT in that the 
GAMMA parameters need to be carefully chosen. Also it uses many random numbers.
One should continue to use SELECT if the permutation acceptance ratio
is greater than .5. 

NHITS should be set to a large value ( on the order of NPARTS^2) because
many permutations are tried out before a successful one is found.  One can
look a "fraction of time in weave" to get some idea of the time spent 
constructing the path.  

GAMMA parameters have a different meaning in PERMUTE and need to be adjusted
by hand.

GAMMA(1,itype) is 
used to select the TYPE of particle to be moved; $\gamma_{1,i}$ is
the probability that type $i$ is moved.  It is only used if
ntypes>1 when we must have $\sum_i \gamma_{1,i} = 1$.  

GAMMA(j,itype) for $j>1$ is the probablity that we continue construction
cycles of length $j$ or longer. The cycle is cutoff at length $m$ if
GAMMA(m+1,itype)=0 or if m=MNMOVERS.  Hence $0 \le \gamma_{j,i} \le 1$. 

Here is a typical setup for generating cycles of length 6 and less:
(MNMOVERS=6)

\begin{verbatim}
GAMMA 1 4HE 1.
GAMMA 2 4HE .98
GAMMA 3 4HE .95
GAMMA 4 4HE  .8
GAMMA 5 4HE  .8
GAMMA 6 4HE  .8
\end{verbatim} 
Here is a fermion setup where only 1, 3 and 5 cycles are generated:
\begin{verbatim} 
GAMMA 1 3HE 1.    ! not really needed if one-component problems.
GAMMA 2 3HE  .95  ! 5% of time try 1 particle moves.
GAMMA 3 3HE  1.   !  no pair exchanges.
GAMMA 4 3HE  .8   ! 20% of time try 3 particle moves.
GAMMA 5 3HE  1.   ! no 4 particle exchanges.
GAMMA 6 3HE  0.   ! no 6 or higher exchanges.
\end{verbatim}


\item 
READPC nblock [ nspb [ levels [ filename
      read packed file and compute energy
      nblock = number of statistical blocks
      nspb= number of steps/block [1]
      levels = time step level [1]
      filename to read configurations [qid.pc]
\end{enumerate}
 
\subsection{ File Naming convention: } 
\begin{itemize}
\item
    .cm  common block (include)
\item
    .dm  is a density matrix file
\item
    .f   fortran source
\item
    .ic  data file containing input coordinates
         (just coordinates at one time slice)
\item
    .in  input file specifying type and length of the run
\item
    .os  diagnostics generated by setup program
\item
    .p   parameters to dimension variables (include)
\item
    .pc packed configuration file = unit 90
\item
    .sc  scalar averages (time history) =unit 62
Source files:
\item
    .f : principle source (recompile if .p files change)
\item
    fnlib.f : 'library' files (do not depend on .p files)
\item
    *set.f  : sources which produce .sy input for various problems
 
Fixed files produced or required by fnpimc:
 
\item
    qid.sy : system file:sets up system and wavefunction
\item
    qid.in : executes the Monte Carlo
\item
    qid.out: output of fnpimc (unit 6)
\item
    qid.sc : scalar averages
\item
    qid.mm : maximum memory used (unit 77)
\item
    qid.gr : g(r)
\item
    qid.sk : s(k)
\item
    qid.gv : rho(k)
\item
    Error messages are written to unit *
\end{itemize}
 

 
\subsection{Data file formats}

A data file consists of a number of tensors.
Two keywords must be defined for each tensor:
Only the RANK and BEGIN commands are required.
\begin{enumerate}
\item RANK (required)
This command gives the rank of the next tensor
and all of its dimensions.
\bv
 RANK nrank n_1,....ni_nrank
\ev
\item
\bv
 BEGIN [ informative labelling possible
\ev
   data starts in the record following the begin
   there must be exactly n1*n2*   nnrank floating words.
 
\item
 CGRID dimension (cvalues(i),i=1,ndimension)
     specifies character labeling of the values in the specified dimen.
\item
 LABEL dimension value
     gives an overall label to a given dimension
\item SIZE
 SIZE indicies minimum maximum
     specifies that x(indices) are between min and max
 
   configuration  data file contaning coordinates or vectors
       ** RANK nrank n1,    nrank; where n1 is spatial dimensionality                                         n2 is the number of particles
                                         n3 number of configuration
       ** SIZE (nrank indices) minimum maximum
             will give the box size in the various dimensions
       **  BEGIN
             all real data

\end{enumerate}
The types of particle, line and plane are implemented as follows.
Types line and plane are implemetented by assuming the line and
plane are parallel to the box axis. Then when vector distances between
particles are found, certain compenents of the vector are zeroed.
Needed in codrift, dsntce and anal and offd. (the calculations are

\section{Examples}
Here is some output for OCset
\begin{verbatim}
 input name of run
bos1    ********************[RUN id]
 input dm file 
C01     *******************
  8 CMD: UNTS H A0
  8 CMD: TYPE p .5
  8 CMD: TYPE p .5
  8 CMD: GRID 150 LINEAR .2 31.
  8 CMD: SQUARER .125 5 3 3 15 14
  8 CMD: POT COUL 24. 1. .001
  8 CMD:POTTAIL    0.22272E-02  0
  8 CMD: RANK 2 150,  1
 input: nparts 
54  *****************************[number of particles]
  input density 
.00023873 *************************
 input crystal number of unit cells in each direction
 crystal:1=sc,2=bcc,3=hcp,4=fcc(3d),3=triangular(2d)
2 3 3 3 ********************************[bcc lattice]
 computing  54 lattice sites  dimensionality 3 number density  0.23873E-03
   body centered cubic lattice
 q displs    0.00000   0.00000   0.00000
 q displs    0.50000   0.50000   0.50000
 npuc  2 nvacancies     0 box size 0.60930E+02 0.60930E+02 0.60930E+02
 number of cells in each direction    3    3    3
 nearest and next nearest neighbor distance   0.17589E+02 0.20310E+02
  possible fits=3
  possible tslice 0.125,  0.25,  0.5,  1.,  2.
  input Temp, tslice 
.0125 .125 ************************************[beta=80, tau=8]
  nslices = 10 temp 1.25E-2
 input: nspins nppss 
0 ****************************[bosons]
  number of k-vectors 2,  9
  input blocks passes, levels for omove 
5 5 3 *******************************[3 levels maximum for omove]
  input blocks, passes, levels, hitsfor select 
0 0 0 0 *****************************[No select]
 STOP 
\end{verbatim}
Here is bos1.sy produced
\begin{verbatim}
    ENORM    0.54000000E+02
 BOXSIZE  0.6092968328E+02  0.6092968328E+02  0.6092968328E+02
     BETA    0.80000000E+02
  NSLICES 10
  NTERM 3 3
TYPE p    0.500000000E+00    0   54 bos1.p.ic     
 POT PAIR  p     p     C01.dm
  GAMMA 3 p     54.
 POT EWALD 0.1554042270264 bos1.lr
    VTAIL   -0.13253382E+00
    PTAIL   -0.31639799E-04
 RESTART
 OMOVE 5 5 3

\end{verbatim}
 
\subsection{Porting}
    Notes for changing to/from various computers

    procedure to move to machine X
Supported machines: C90, sun, sgi, hp rs6000, 



  Edit upi.p for dimensions (number of particles etc.)


\end{document}
